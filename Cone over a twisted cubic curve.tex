\documentclass{amsart}[12pt]

\usepackage[cmtip,all]{xy}
\usepackage[utf8]{inputenc}
\usepackage{amsmath, amscd}
\usepackage{mathrsfs}
\usepackage{amssymb}
\usepackage{mathtools}
\usepackage{url}
\usepackage[top=1.3in, bottom=1.3in, left=1.3in, right=1.3in]{geometry}
\usepackage{pxfonts}
\usepackage{tikz}
\usepackage{tikz-cd}
\usetikzlibrary{arrows,chains,matrix,positioning,scopes}
\tikzset{join/.code=\tikzset{after node path={%
\ifx\tikzchainprevious\pgfutil@empty\else(\tikzchainprevious)%
edge[every join]#1(\tikzchaincurrent)\fi}}}
\tikzset{>=stealth',every on chain/.append style={join},
         every join/.style={->}}
\usepackage{color}


\DeclareMathOperator{\coker}{coker}
\DeclareMathOperator{\im}{im}


\newtheorem{theorem}{Theorem}[section]
\newtheorem{lemma}[theorem]{Lemma}
\newtheorem{cor}[theorem]{Corollary}
\newtheorem{prop}[theorem]{Proposition}
\theoremstyle{definition}
\newtheorem{defn}[theorem]{Definition}
\newtheorem{eg}[theorem]{Example}
\newtheorem{ex}[theorem]{Exercise}
\newtheorem{claim}[theorem]{Claim}
\newtheorem{obs}[theorem]{Observation}

\theoremstyle{remark}
\newtheorem{rmk}[theorem]{Remark}

\numberwithin{equation}{section}

%    Absolute value notation
\newcommand{\abs}[1]{\lvert#1\rvert}
\newcommand{\To}{\longrightarrow}


% ----------------------------------------------------------------
\begin{document}

\title[The Hilbert scheme of points on the cone over a twisted cubic]{The Hilbert scheme of points on the cone over a twisted cubic}%
%\author{Xudong Zheng}%
\address{University of Illinois at Chicago, 851 South Morgan Street, Science and Engineering Offices m/c249, Chicago, IL 60607 USA}%
\email{xzheng20@uic.edu}%

\thanks{}%
\subjclass{}%

\keywords{}%

\date{\today}%
%\dedicatory{}%
%\commby{}%
% ----------------------------------------------------------------

\maketitle

\section{Set up}
\begin{align*}
S & = \mathbb{C}[x, y, z, w] \\
I & = \langle xz - y^2, xw - yz, yw - z^2 \rangle = \textrm{ ideal of the cone over the twisted cubic curve}\\
R & = S / I \\
X & = \mathrm{Spec}(R) = \textrm{the cone}\\
\end{align*}

In \cite[Proposition 5.1]{8points}, Cartwright, Erman, Velasco, and Viray find a second irreducible component of $\mathrm{Hilb}^8(\mathbb{A}^4)$ of dimension at most 25 by computing the Zariski tangent space of $\mathrm{Hilb}^8(\mathbb{A}^4)$ at a closed point corresponding to a length $8$ subscheme that is defined by 7 quadratic forms in $\mathbb{A}^4$. Explicitly, let $Z$ be defined by 
\[
J = \langle x^2, xy y^2, z^2, zw, xw + yz \rangle,
\]
then the tangent space to the Hilbert scheme is $25$ dimensional.

\section{Statement}

\begin{prop}
The Hilbert scheme $\mathrm{Hilb}^8(X)$ is reducible.
\begin{proof}
We modify the ideal in their example:
\[
J_0 = \langle x^2, xy, xz - y^2, xw - yz, yw - z^2, zw, w^2 \rangle.
\]
In particular, note that the scheme $Z = \mathrm{Spec}(S/J_0)$ still has length $8$ since $\langle x, y, z, w \rangle^3 \subset J_0$, and it is scheme-theoretically embedded in the singular surface $X$ since $I \subset J_0$.

Now we compute the Zariski tangent space of $\mathrm{Hilb}^8(\mathbb{A}^4)$ at $[Z]$: choose a $\mathbb{C}$-linear space basis of $\mathcal{O}_Z$ as $\{1, x, y, z, w, xw, xz, yw\}$. Then an $S$-linear homomorphism $\phi: J_0 \to S/J_0$ can be represented by a table of the form
\begin{center}
      \begin{tabular}[b]{ c | c c c c c c c}
    & $x^2$ & $xy$ & $xz - y^2$ & $xw - yz$ & $yw - z^2$ & $zw$ & $w^2$ \\ \hline
    1 &  0 & 0 & 0 & 0 & 0 & 0 & 0  \\
    $x$ & $b_2 + c_3$ & 0 & $c_1$ & 0 & 0 & $f_1$ & 0 \\
    $y$ & $a_2$ & $b_2$ & $-c_3$ & $c_1 - 2c_3$ & 0 & $f_2$ & $f_1$ \\ 
    $z$ & $a_3$ & $a_2$ & $c_3$ & 0 & $-2c_3$ & $f_3$ & $f_2$ \\
    $w$ & 0 & $a_3$ & 0 & $-c_3$ & $c_3$ & $c_1 - 2c_3$ & $f_3$ \\
    $xw$ & * & * & * & * & * & * & * \\
    $xz$  & * & * & * & * & * & * & * \\
    $yw$  & * & * & * & * & * & * & * \\
  \end{tabular}
\end{center}
where the *'s and $a_2, b_2, a_3, c_1, c_3, f_1, f_2, f_3$ in the table can take arbitrary values. The total number of such free entries in the table counts the dimension $\dim T_{[Z]}\mathrm{Hilb}^8(\mathbb{A}^4) = 21 + 8 = 29 < 4 \times 8 = 32$. We conclude that the closed point $[Z]$ does not lie on the main component of $\mathrm{Hilb}^8(\mathbb{A}^4)$ (not even the intersection of the main component with other components). But $[Z] \in \mathrm{Hilb}^8(X)$ as a closed point, and if $\mathrm{Hilb}^8(X)$ were irreducible then $\mathrm{Hilb}^8(X)$ is contained in the main component of $\mathrm{Hilb}^8(\mathbb{A}^4)$. This shows that $\mathrm{Hilb}^8(X)$ is reducible.
\end{proof}
\end{prop}
\bibliographystyle{amsalpha}
\begin{thebibliography}{CEVV}
\raggedright

\bibitem[CEVV]{8points}
D. Cartwright, D. Erman, M. Velasco, B. Viray. Hilbert schemes of 8 points, \textit{Algebra and Number Theory}, 3, 763-795, 2009.


\end{thebibliography}
\end{document}
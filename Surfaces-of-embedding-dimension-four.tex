\documentclass{amsart}[12pt]

\usepackage[cmtip,all]{xy}
\usepackage[utf8]{inputenc}
\usepackage{amsmath}
\usepackage{mathrsfs}
\usepackage{amssymb}
\usepackage{mathtools}
\usepackage{url}
\usepackage[top=1.3in, bottom=1.3in, left=1.3in, right=1.3in]{geometry}
\usepackage{pxfonts}
\usepackage{tikz-cd}
\usetikzlibrary{matrix,arrows}
\usepackage{hyperref}
\usepackage[boxsize=2em]{ytableau}
\usepackage{ytableau,varwidth}
\usetikzlibrary{calc}
\usepackage{eepic}
\usepackage{mathbbol}
\ytableausetup{centertableaux}
\usepackage{color}


\newtheorem{theorem}{Theorem}[section]
\newtheorem{lemma}[theorem]{Lemma}
\newtheorem{cor}[theorem]{Corollary}
\newtheorem{prop}[theorem]{Proposition}
\theoremstyle{definition}
\newtheorem{defn}[theorem]{Definition}
\newtheorem{eg}[theorem]{Example}
\newtheorem{ex}[theorem]{Exercise}
\newtheorem{fact}[theorem]{Fact}
\newtheorem{ob}[theorem]{Observation}
\newtheorem{claim}[theorem]{Claim}
\newtheorem{question}[theorem]{Question}
\newtheorem{obs}[theorem]{Observation}

\linespread{1.5}

\theoremstyle{remark}
\newtheorem{rmk}[theorem]{Remark}

\numberwithin{equation}{section}

%    Absolute value notation
\newcommand{\abs}[1]{\lvert#1\rvert}
\newcommand{\To}{\longrightarrow}
\newcommand*{\sheafhom}{\mathscr{H}\kern -.5pt om}

% ----------------------------------------------------------------
\begin{document}

\title[Surfaces of embedding dimension three and four]{Surfaces of embedding dimension three and four}%
%\author{Xudong Zheng}%
%\address{University of Illinois at Chicago, 851 South Morgan Street, Science and Engineering Offices m/c249, Chicago, IL 60607 USA}%
%\email{xzheng20@uic.edu}%

\maketitle
\date{\today}%

\section{Introduction}
Let $X$ be a quasi-projective surface over an algebraically closed field $\mathbb{k}$ of characteristic 0, and let $d$ be a positive integer. The Hilbert scheme of $d$ points, denoted by $\mathrm{Hilb}^d(X)$, is the moduli space of length $d$ subschemes of $X$. The aim of this paper is to find conditions for the singularities of $X$ such that $\mathrm{Hilb}^d(X)$ is reducible for $d$ sufficiently large.

If $X$ is the affine cone over a smooth projective curve of degree at least 5, then $\mathrm{Hilb}^d(X)$ is reducible for large enough $d$. It is easy to see (Lemma \ref{easy}) that if either there is a singular point where $X$ has embedding dimension at least 5 or the singular locus of $X$ is not isolated, then $\mathrm{Hilb}^d(X)$ is reducible already for $d = 2$. On the other hand, if $X$ has at worst Kleinian singularities, then $\mathrm{Hilb}^d(X)$ is irreducible. 

Emsalem-Iarrobino (\cite{EI78}) constructed a 25-dimensional non-smoothable component of $\mathrm{Hilb}^8(\mathbb{A}^4)$, which was later proven to be the only non-smoothable component (\cite{CEVV}). The idea of Emsalem-Iarrobino is to show that the Zariski tangent space at a general point of such a component has dimension 25, smaller than $32 = \dim \mathrm{Hilb}^8(\mathbb{A}^4)$. The ideal of a general length 8 non-smoothable subscheme of $\mathbb{A}^4$ from this component is homogeneous, generated by seven quadratic forms with only linear syzygies. Then the Zariski tangent space to $\mathrm{Hilb}^8(\mathbb{A}^4)$ at such a point is naturally graded, and it amounts to calculate the dimension of each graded summand of the tangent space. Moreover, the intersection of the two components of $\mathrm{Hilb}^8(\mathbb{A}^4)$ is a divisor of the non-smoothable component, which can be described explicitly. Hence non-smoothability of length $8$ subschemes of $\mathbb{A}^4$ with a fixed support gives an open condition on $\mathbb{G}(6, \mathbb{P}(H^0(\mathbb{P}^3, \mathcal{O}_{\mathbb{P}^3}(2))))$. The fact that the ideal of a non-smoothable length $8$ subscheme of $\mathbb{A}^4$ is quadratic homogeneous motivates further investigation into the irreducibility of the Hilbert scheme of points on singular surfaces where the tangent cones at the singular loci are defined by quadrics. 

Suppose $\abs{L} \subset \mathbb{P}(H^0(\mathbb{P}^3, \mathcal{O}_{\mathbb{P}^3}(2)))$ is a sublinear system such that $\mathrm{Bs}(L)$ is one-dimensional. For example, $\mathrm{Bs}(L)$ is a smooth elliptic curve in $\mathbb{P}^3$ with $\dim \abs{L} = 1$ or $\mathrm{Bs}(L)$ is a twisted cubic curve in $\mathbb{P}^3$ with $\dim \abs{L} = 2$. If there exists $\abs{\tilde{L}} \in \mathbb{G}(6, \mathbb{P}(H^0(\mathbb{P}^3, \mathcal{O}_{\mathbb{P}^3}(2))))$ such that $\abs{L} \subset \abs{\tilde{L}}$ and that $\abs{\tilde{L}}$ satisfies the non-smoothability condition. Then there exist non-smoothable length $8$ schemes of $\mathbb{A}^4$ that is scheme-theoretically embedded in the singular surface, the affine cone $X \coloneqq C(\mathrm{Bs}(L))$ over the base locus of $\abs{L}$. Then $\mathrm{Hilb}^8(X)$ will be reducible. For example, since the twisted cubic curve is determinantal, for any net of quadrics $\abs{L}$ defining a twisted cubic curve one can find such an $\abs{\tilde{L}}$, and the Hilbert scheme of 8 points on the cone over any twisted cubic is reducible. 


\section{Preliminaries}
\begin{lemma}\label{easy} Suppose $X$ is a quasi-projective surface with singular locus $X_{sing}$, and suppose $p \in X_{sing}$ is a closed point. Then $\mathrm{Hilb}^d(X)$ is reducible for any $d \geq 2$ in the following two cases.
\begin{itemize}
\item[(1)] $(X, p)$ is the germ with embedding dimension at least 5. 
\item[(2)] $X_{sing}$ contains a one-dimensional component. 
\end{itemize}
\begin{proof}
In both cases, the reducibility of $\mathrm{Hilb}^2(X)$ implies the reducibility of $\mathrm{Hilb}^d(X)$ for any $d \geq 2$.

(1) The projectivized Zariski tangent space $\mathbb{P}(T_p(X))$ is isomorphic to the punctual Hilbert scheme $\mathrm{Hilb}^2(X, p)$, which is at least 4 dimensional. On the other hand, the locus of $\mathrm{Hilb}^2(X)$ of smoothable length 2 subschemes of $X$ is 4 dimensional. Therefore, a general point of $\mathrm{Hilb}^2(X, p)$ corresponds to a non-smoothable length 2 subscheme.

(2) If $X_{sing}$ has a one-dimensional component $X_1$, then $X$ cannot be Cohen-Macaulay, and hence cannot have hypersurface singularity. Then the embedding dimension of $X$ at any point in $X_1$ is at least 4. Consequently, $\mathrm{Hilb}^2(X_1)$ is at least 4 dimensional, whose general point parametrizes a non-smoothable length 2 subscheme of $X$.
\end{proof} 
\end{lemma}



From now on, we assume all surfaces have only isolated singularities with embedding dimension at most 4. 

\subsection{A non-smoothable component of the Hilbert scheme of 8 points in $\mathbb{A}^4$}

Denote by $S = \mathbb{C}[x_1, \dots, x_n]$ the polynomial ring in $n$ variables, and by $S^* = \mathbb{C}[y_1, \dots, y_n]$ the graded $S$-module whose module structure is given by formal differentiation. That is, for any $f \in S^*$ define $x_i \cdot f = \dfrac{\partial f}{\partial y_i}$ for $i = 1, \dots, n$; this extends to higher order partial differentiations. In particular, an $S$-submodule of $S^*$ is closed under partial differentiation. 

For any $j \geq 0$, denote by $S_j = H^0(\mathbb{P}^{n - 1}, \mathcal{O}_{\mathbb{P}^{n - 1}}(j))$ the space of polynomials of degree $j$, and by $S_j^*$ the degree $j$ summand of $S^*$. Then there is a bilinear pairing 
\begin{equation}\label{pairing}
\Psi: S_j \times S_j^* \to S_0^* \cong \mathbb{C}.
\end{equation}
If $V$ is a subspace of $S_j$ for some $j$, then there exists a unique subspace $V^{\perp}$ of $S_j^*$ such that $\dim V + \dim V^{\perp} = \dim S_j$ and $\Psi(v, f) = 0$ for any $v \in V$ and $f \in V^{\perp}$. Suppose $I$ is a homogeneous ideal of $S$. Define the \textit{Macaulay inverse system} of $I$, denoted by $I^{\perp} = \oplus_j(I^{\perp})_j$, as the graded $S$-submodule of $S^*$ such that $(I^{\perp})_j \cong (I_j)^{\perp}$ for any $j$.

Specializing to the case of $\mathbb{C}^4$, let $S = \mathbb{C}[x_1, \dots, x_4]$. Suppose $F_1, F_2, F_3 \in S_2^*$ are three quadrics in $y_1, y_2, y_3$, and $y_4$. Let $A_i$ be the symmetric $4 \times 4$ matrix associated to $F_i$ for $i = 1, 2, 3$. The \textit{Salmon-Turnbull Pfaffian} is the Pfaffian of the skew-symmetric matrix
\[
\begin{bmatrix}
0 & A_1 & -A_2 \\ -A_1 & 0 & A_3 \\ A_2 & -A_3 & 0
\end{bmatrix}.
\]

\begin{lemma}
In the Pl\"ucker coordinates of $Gr(3, S_2^*)$, the Salmon-Turnbull Pfaffian is irreducible. Moreover, the Salmon-Turnbull Pfaffian is the unique quadric $GL(4, \mathbb{C})$-invariant polynomial in the Pl\"ucker coordinates of $Gr(3, S_2^*)$.
\end{lemma}

\begin{theorem} The Hilbert scheme $\mathrm{Hilb}^8(\mathbb{A}^4)$ has two irreducible components $R_8^4$ and $G_8^4$.
\begin{itemize}
\item[(1)] The non-smoothable component $G_8^4$ is 25 dimensional, parameterizing subschemes supported at a single point with local Hilbert function $(1, 4, 3)$, and $G_8^4 \cong \mathbb{A}^4 \times G_0$, where $G_0 \cong Gr(3, S_2^*)$ parametrizes length $8$ schemes with local Hilbert function $(1, 4, 3)$ supported at a fixed point.
\item[(2)] Let $W_0$ be the intersection of $G_0$ with the principal component $R_8^4$. Then $W_0$ is a prime divisor of $G_0$, defined by the Salmon-Turnbull Pfaffian in $G_0$.
\end{itemize}
\end{theorem}

\begin{cor}
Suppose $X$ is a quasi-projective surface with only isolated singularities and any the embedding dimension of $X$ at any singular point is 4. Assume that the tangent cone of $X$ at a singular point is generated in degree 3 and higher. Then the Hilbert scheme of points $\mathrm{Hilb}^8(X)$ is reducible. 
\end{cor}
\begin{proof}
Let $p \in X$ be a singular point such that the tangent cone $TC_p(X)$ has homogeneous ideal generated in degree 3 and higher. Then a non-smoothable length 8 subscheme of $\mathbb{A}^4$ supported at $p$ is also a subscheme of $X$. Then there exists a component of $\mathrm{Hilb}^8(X)$ of non-smoothable schemes. 
\end{proof}

\subsection{Linear Systems of Quadrics}
Consider the space of quadrics $\mathbb{P}(S_2) \cong \mathbb{P}^{N}$ in $\mathbb{P}^{n - 1}$ where $N = {n + 1 \choose 2} - 1$. The discriminant locus is the hypersurface $\Delta \subset \mathbb{P}^{N}$ of degree $n$: 
\[
\Delta = \left\{ [\lambda_0 \colon \dots \colon \lambda_{N}] \in \mathbb{P}^{N} \mid Q_i \in S_2, \det\left(\sum_{i = 0}^{N} \lambda_i Q_i\right) = 0\right\}.
\]
The action of $\mathrm{GL}(n, \mathbb{C})$ on $S_2$ via linear transformations of the variables extends to an action on $\mathbb{P}^{N}$. The coefficients of the discriminant hypersurface $\Delta$ are invariant under $\mathrm{SL}(n, \mathbb{C})$, restricted from $\mathrm{GL}(n, \mathbb{C})$. Suppose $\abs{L} \cong \mathbb{P}^{r} \subset \mathbb{P}^{N}$ is a sublinear system of quadrics for some integer $N \geq r \geq 1$. The discriminant restricts to $\abs{L}$ as a hypersurface $\Delta_{\abs{L}} \subset \abs{L}$, and $\Delta_{\abs{L}}$ is called the discriminant of $\abs{L}$. If $Q_0, \dots, Q_{r} \in \abs{L}$ is any collection of distinct elements of $\abs{L}$ which spans $\abs{L}$, i.e., any element of $\abs{L}$ is expressible as $\sum_{i = 0}^{r} \lambda_i Q_i$ for some $[\lambda_0 \colon \dots \colon \lambda_{r}] \in \mathbb{P}^{r}$, write $\abs{L} = \left(\sum_{i = 0}^{r} \lambda_i Q_i\right)$ and call $\{Q_0, \dots, Q_{r}\}$ a set of basis of $\abs{L}$. We say that the linear system $\abs{L} = \left\{\sum_{i = 0}^{r} \lambda_i Q_i\right\}$ \textit{fails} to be stable (resp. semistable) if there exists a semisimple $1$-parameter subgroup $\theta(t)$ of $\mathrm{SL}(n, \mathbb{C})$, depending on a multiplicative parameter $t$ such that each $\theta(t)Q_i$ is bounded as $t \to \infty$ (resp. tends to $0$ as $t \to \infty$) for $i = 0, \dots, r$.

\begin{theorem}\cite[Theorem 0.1]{W78} Let $\abs{L} \subset \mathbb{P}^{N}$ be an $(r - 1)$-dimensional sublinear system of quadrics.
\begin{enumerate}
\item The system $\abs{L}$ is unstable if and only if for some choice of coordinates $(x_1, \dots, x_n)$ so that $\abs{L} = \left(\sum_{i = 0}^{r - 1} \lambda_i Q_i\right)$ with $Q_i = \sum_{j, k}a_{j, k}^{(i)}x_jx_k$, and some integer $s$, $1 \leq s \leq n/2$ such that
\[
a_{j, k}^{(i)} = 0
\] 
for any $0 \leq i \leq r - 1$, $1 \leq j \leq s$, and $1 \leq k \leq n - s$.
\item It fails to be semistable if and only if there exist a coordinate system $(x_1, \dots, x_n)$ and an integer $s$ with $1 \leq s \leq (n + 1)/2$ such that 
\[
a_{j, k}^{(i)} = 0
\]
for any $0 \leq i \leq r - 1$, $1 \leq j \leq s$, and $1 \leq k \leq n + 1 - s$.
\end{enumerate}
\end{theorem}

From now on, we specialize to $n = 4$. 

\begin{eg}
A pencil $L$ of quadrics in $\mathbb{P}^{3}$ consists of singular quadrics if and only if:
\begin{enumerate}
\item The quadrics in $L$ have a common singular point; or
\item Restricted to a common plane, quadrics in $L$ contain a double line. The general such $L$ is, up to projectivity, $\lambda (x_3^2 + 2x_2x_4) +  \mu x_1x_4$.
\end{enumerate}
\end{eg}


\begin{eg}
A net of quadrics $\{Q_0, Q_1, Q_2\}$ in $\mathbb{P}^3$ fails to be stable if and only if there exists a system of coordinates such that the matrices of $Q_i$ are all of the form 
\[
(s = 1): \begin{bmatrix} 0 & 0 & 0 & * \\ 0 & * & * & * \\ 0 & * & * & * \\ * & * & * & *\end{bmatrix}; \quad (s = 2): \begin{bmatrix} 0 & 0 & * & * \\ 0 & 0 & * & * \\ * & * & * & * \\ * & * & * & * \end{bmatrix}
\]
It fails to be semistable if and only if their matrices can be reduced to
\[
(s = 1): \begin{bmatrix} 0 & 0 & 0 & 0 \\ 0 & * & * & * \\ 0 & * & * & * \\ 0 & * & * & *\end{bmatrix}; \quad (s = 2): \begin{bmatrix} 0 & 0 & 0 & * \\ 0 & 0 & 0 & * \\ 0 & 0 & * & * \\ * & * & * & * \end{bmatrix}
\]
\end{eg}

\section{Pencils of Quadrics}
\subsection{Normal forms of pencils of quadrics in $\mathbb{P}^3$}
The exposition of this subsection follows \cite{HP52} closely. Let $\abs{L} = (\lambda Q_0 + \mu Q_1)$ be a pencil of quadrics. If a general member of the pencil is non-singular, then the discriminant is not identically zero. The classification is reduced to the classification of multiplicities of elementary divisors of the discriminant quartic $\Delta_{\abs{L}} = \det (\lambda A_0 + \mu A_1)$ where $A_i$ is the symmetric matrix of $Q_i$ for $i = 0$ and $1$. If the pencil consists of singular members, the classification is determined by the rank of a general member of the pencil. 

\begin{theorem}\cite[Chap. XIII, Sec. 10, Theorem I, II, III]{HP52} Let $\abs{L} = (Q_0, Q_1)$ be a pencil of quadrics in $\mathbb{P}^3$. 
\begin{itemize}
\item[(I)] Suppose there is at least one non-singular member in $\abs{L}$. 
\end{itemize}
\begin{itemize} 
\item[(a).] Two such pencils $\abs{L_1}$ and $\abs{L_2}$ are projectively equivalent if and only if their discriminants $\Delta_{\abs{L_1}}$ and $\Delta_{\abs{L_2}}$ have the same, up to permutation, elementary divisors.
\item[(b).] There exist projective transformations so that $Q_0$ and $Q_1$ can be reduced to their normal forms. 
\begin{enumerate}
\item[(I. 1)] The discriminant $\Delta_{\abs{L}}$ has 4 distinct divisors. Then $Q_0$ and $Q_1$ can be simultaneously diagonalized.
\[
Q_0 = x_1^2 + x_2^2 + x_3^2 + x_4^2, \quad Q_1 = a_1x_1^2 + a_2x_2^2 + a_3x_3^2 + a_4x_4^2
\]
for $(a_1, a_2, a_3, a_4) \in \mathbb{C}^4$.
\item[(I. 2)] $\Delta_{\abs{L}}$ has divisors of type $(2, 1, 1)$ (one double factor). 
\[
Q_0 = 2x_1x_2 + x_3^2 + x_4^2, \quad Q_1 = 2a_1x_1x_2 + x_2^2 + a_2x_3^2 + a_3x_4^2
\]
for $(a_1, a_2, a_3) \in \mathbb{C}^3$.
\item[(I. 3)] $\Delta_{\abs{L}}$ has divisors of type $(2, 2)$ (two double factors). 
\[
Q_0 = 2x_1x_3 + x_2^2 + x_4^2, \quad Q_1 = 2a_1x_1x_3 + a_1x_2^2 + a_2x_4^2 + 2x_2x_3
\]
for $(a_1, a_2) \in \mathbb{C}^2$.
\item[(I. 4)] $\Delta_{\abs{L}}$ has divisors of type $(3, 1)$ (one triple factor). 
\[
Q_0 = 2x_1x_2 + 2x_3x_4, \quad Q_1 = 2a_1x_1x_2 + x_2^2 + 2a_2x_3x_4 + x_4^2
\]
for $(a_1, a_2) \in \mathbb{C}^2$.
\end{enumerate}
\end{itemize}
\begin{itemize}
\item[(II)] If a general member of $\abs{L}$ has rank 3.
\begin{enumerate}
\item[(II. 1)] $Q_0$ and $Q_1$ are two cones. There are two cases.
\begin{align*}
& Q_0 = x_1^2 + x_2^2 + x_3^2, \quad Q_1 = a_1x_1^2 + a_2x_2^2 + a_3x_3^2. \\
& Q_0 = 2x_1x_2 + x_3^2, \quad Q_1 = 2a_1x_1x_2 + x_2^2 + a_2x_3^2. 
\end{align*}
\item[(II. 2)] A plane-pair and a cone.
\[
Q_0 = 2x_1x_2 + x_3^2, \quad Q_1 = 2x_2x_3. 
\]
\item[(II. 3)] Two plane pairs. There are two cases.
\begin{align*}
& Q_0 = 2x_1x_2 + a_1x_3^2, \quad Q_1 = x_2^2 + x_3^2.\\
& Q_0 = x_1x_2, \quad Q_1 = x_2x_3. 
\end{align*}
\end{enumerate}
\item[(III)] If a general member of $\abs{L}$ has rank 2.
\begin{enumerate}
\item[(III. 1)] Two plane-pairs. There are two cases.
\begin{align*}
& Q_0 = x_1^2 + x_2^2, \quad Q_1 =  a_1x_1^2 + a_2x_2^2.\\
& Q_0 = x_1^2 + x_2^2, \quad Q_1 =  2x_1x_2. 
\end{align*}
\item[(III. 2)] A plane-pair and a double plane.
\[
Q_0 = 2x_1x_2, \quad Q_1 = x_1^2.
\]
\item[(III. 2)] Two double planes.
\[
Q_0 = x_1^2, \quad Q_1 = x_2^2.
\]
\end{enumerate}
\end{itemize}
\end{theorem}

\subsection{Base loci of pencils of quadrics in $\mathbb{P}^3$}
To describe the base loci of pencils of quadrics, it is necessary to refine the list of the normal forms above. For example, the case (I. 1) where the discriminant quartic has 4 distinct divisors is further divided into 4 subcases. All of the coefficients appearing in the list below are assumed to be non-zero, avoiding duplications with degenerate cases.

I. The pencil is non-singular.
\begin{itemize}
\item[(I. 1)] The discriminant $\Delta_{\abs{L}}$ has 4 distinct divisors. 
\begin{enumerate}
\item[(1)] 
\[
Q_0 = x_1^2 + x_2^2 + x_3^2 + x_4^2, \quad Q_1 = a_1x_1^2 + a_2x_2^2 + a_3x_3^2 + a_4x_4^2,
\]
where $a_i$ are distinct for $i = 1, \dots, 4$. Then $\mathrm{Bs}(L)$ is an irreducible quartic curve.
\item[(2)]
\[
Q_0 = x_1^2 + x_2^2 + x_3^2 + x_4^2, \quad Q_1 = a_1x_1^2 + a_1x_2^2 + a_3x_3^2 + a_4x_4^2,
\]
where $a_i$ are distinct for $i = 1, 3, 4$. Then $\mathrm{Bs}(L)$ is the union of two plane conics, which is the intersection of $Q_1$ with the two planes defined by the two linear factors of $(a_3 - a_1)x_3^2 + (a_4 - a_1)x_4^2$.
\item[(3)]
\[
Q_0 = x_1^2 + x_2^2 + x_3^2 + x_4^2, \quad Q_1 = a_1x_1^2 + a_1x_2^2 + a_3x_3^2 + a_3x_4^2,
\]
where $a_1 \neq a_3$. Then $\mathrm{Bs}(L)$ consists of four lines, the intersection of the two plane-pairs $x_1^2 + x_2^2$ and $x_3^2 + x_4^2$.
\item[(4)]
\[
Q_0 = x_1^2 + x_2^2 + x_3^2 + x_4^2, \quad Q_1 = a_1x_1^2 + a_1x_2^2 + a_1x_3^2 + a_4x_4^2,
\]
where $a_1 \neq a_4$. Then $\mathrm{Bs}(L)$ is the non-reduced double conic over $x_1^2 + x_2^2 + x_3^2$ with $x_4^2 = 0$.
\end{enumerate}

\item[(I. 2)] $\Delta_{\abs{L}}$ has divisors of type $(2, 1, 1)$ (one double factor). The two quadrics have the same tangent plane $x_2 = 0$ at $[1\colon 0 \colon 0 \colon 0]$.
\begin{enumerate}
\item[(1)] 
\[
Q_0 = 2x_1x_2 + x_3^2 + x_4^2, \quad Q_1 = 2a_1x_1x_2 + x_2^2 + a_2x_3^2 + a_3x_4^2,
\]
where $a_i$ are distinct for $i = 1, 2, 3$. Then $\mathrm{Bs}(L)$ is a quartic curve with an embedded point at $[1\colon 0 \colon 0 \colon 0]$. Eliminating $x_1$ we obtain a plane conic $\bar{Q}_1 = x_2^2 + (a_2 - a_1)x_3^2 + (a_3 - a_1)x_4^2$. If $x_2 \neq 0$, then a zero locus of $\bar{Q}_1$ gives rise to a point in $\mathrm{Bs}(L)$, which is a quartic. Lastly, the quartic has a nodal singular point at $[1\colon 0 \colon 0 \colon 0]$.
\item[(2)] 
\[
Q_0 = 2x_1x_2 + x_3^2 + x_4^2, \quad Q_1 = 2a_1x_1x_2 + x_2^2 + a_1x_3^2 + a_3x_4^2,
\]
where $a_1 \neq a_3$. Then $\mathrm{Bs}(L)$ is the union of two plane conics intersecting at 1 point of multiplicity 2. The two conics are given by $Q_0$ and the plane-pair $x_2^2 + (a_3 - a_1)x_4^2$. They intersect at $x_2 = x_4 = 0$ on $Q_0$, hence has multiplicity 2.
\item[(3)] 
\[
Q_0 = 2x_1x_2 + x_3^2 + x_4^2, \quad Q_1 = 2a_1x_1x_2 + x_2^2 + a_2x_3^2 + a_2x_4^2,
\]
where $a_1 \neq a_2$. Then $\mathrm{Bs}(L)$ is the union of a pair of lines and a plane conic, which is not coplanar with the lines and does not go through the intersection of the lines. The two lines are the intersection of $Q_0$ with the common tangent plane of $Q_1$ and $Q_0$ at $[1\colon 0 \colon 0 \colon 0]$. The conic is the intersection of $Q_0$ with the plane $2(a_1 - a_2)x_1 + x_2$ which does not go through $[1\colon 0 \colon 0 \colon 0]$.
\item[(4)] 
\[
Q_0 = 2x_1x_2 + x_3^2 + x_4^2, \quad Q_1 = 2a_1x_1x_2 + x_2^2 + a_1x_3^2 + a_1x_4^2.
\]
Then $\mathrm{Bs}(L)$ is the union of a pair of double lines.
\end{enumerate}

\item[(I. 3)] $\Delta_{\abs{L}}$ has divisors of type $(2, 2)$ (two double factors). 
\begin{enumerate}
\item[(1)] 
\[
Q_0 = 2x_1x_3 + x_2^2 + x_4^2, \quad Q_1 = 2a_1x_1x_3 + a_1x_2^2 + a_2x_4^2 + 2x_2x_3,
\]
where $a_1 \neq a_2$. Then $\mathrm{Bs}(L)$ is a cuspidal quartic curve. The quartic curve is singular at $[1\colon 0 \colon 0 \colon 0]$, which is a cusp.
\item[(2)] 
\[
Q_0 = 2x_1x_3 + x_2^2 + x_4^2, \quad Q_1 = 2a_1x_1x_3 + a_1x_2^2 + a_1x_4^2 + 2x_2x_3.
\]
Then $\mathrm{Bs}(L)$ is the union of a smooth conic and a pair of lines intersecting on the conic. The conic is given by $x_2 = 0$ and $Q_0$, and the pair of lines is given by $x_2^2 + x_4^2$ and $Q_0$, whose intersection is $[1\colon 0 \colon 0 \colon 0]$ lying on the conic.
\end{enumerate}

\item[(I. 4)] $\Delta_{\abs{L}}$ has divisors of type $(3, 1)$ (one triple factor). 
\begin{enumerate}
\item[(1)] 
\[
Q_0 = 2x_1x_2 + 2x_3x_4, \quad Q_1 = 2a_1x_1x_2 + x_2^2 + 2a_2x_3x_4 + x_4^2,
\]
where $a_1 \neq a_2$. Then $\mathrm{Bs}(L)$ is the union of a cubic curve and a line. The line is $x_2 = x_4 = 0$, and the cubic curve intersects the line at $[1\colon 0 \colon 0 \colon 0]$ and $[0 \colon 0 \colon 1 \colon 0]$.
\item[(2)] 
\[
Q_0 = 2x_1x_2 + 2x_3x_4, \quad Q_1 = 2a_1x_1x_2 + x_2^2 + 2a_1x_3x_4 + x_4^2.
\]
Then $\mathrm{Bs}(L)$ is the union of a double line and two reduced lines. The line counted twice is $x_2 = x_4 = 0$. Each plane in the plane-pair $x_2^2 + x_4^2$ meets $Q_0$ along the line $x_2 = x_4 = 0$ together with another line. 
\end{enumerate}
\end{itemize}

II. A general member of $\abs{L}$ has rank 3.


\begin{itemize}
\item[(II. 1)] $Q_0$ and $Q_1$ are two cones. 
\begin{enumerate}
\item[(1)] 
\[ 
Q_0 = x_1^2 + x_2^2 + x_3^2, \quad Q_1 = a_1x_1^2 + a_2x_2^2 + a_3x_3^2
\]
\begin{itemize}
\item[(a)] If $a_i$ are all distinct for $i = 1, 2, 3$, then the base locus consists of 4 distinct lines through the common vertex.
\item[(b)] If $a_1 = a_2 \neq a_3$, then $Q_0$ and $Q_1$ can be reduced to a double plane $x_3^2 = 0$ and a plane-pair $x_1^2 + x_2^2 = 0$, which intersect along two double lines.
\end{itemize}
\item[(2)] 
\[ 
Q_0 = 2x_1x_2 + x_3^2, \quad Q_1 = 2a_1x_1x_2 + x_2^2 + a_2x_3^2
\]
\begin{itemize}
\item[(a)] If $a_1 \neq a_2$, then the base locus consists of two reduced lines $2(a_1 - a_2)x_1 + x_2 = 0, x_2^2 + (a_2 - a_1)x_3^2 = 0$ and a double line $x_2 = 0, x_3^2 = 0$ all through the common vertex.
\item[(b)] If $a_1 = a_2$, then the base locus is a line of multiplicity 4.
\end{itemize}
\end{enumerate}

\item[(II. 2)] A plane-pair and a cone.
\[
Q_0 = 2x_1x_2 + x_3^2, \quad Q_1 = 2x_2x_3
\]
The base locus consists of a reduced line $x_1 = x_3 = 0$ and a triple line supported on $x_2 = x_3 = 0$.

\item[(II. 3)] Two plane pairs. 
\begin{enumerate}
\item[(1)] 
\[
Q_0 = 2x_1x_2 + a_1x_2^2, \quad Q_1 = x_2^2 + x_3^2
\]
The base locus consists of two reduced lines and a double line (similar to II. 1.2(a)).
\item[(2)]  
\[
Q_0 = x_1x_2, \quad Q_1 = x_2x_3
\]
The base locus is their common linear component $x_2 = 0$.
\end{enumerate}
\end{itemize}

III. If a general member of $\abs{L}$ has rank 2. 

\begin{itemize}
\item[(III. 1)] Two plane-pairs.
\begin{enumerate}
\item[(1)] 
\[ 
Q_0 = x_1^2 + x_2^2, \quad Q_1 =  a_1x_1^2 + a_2x_2^2
\]
If $a_1 \neq a_2$, the base locus is a line of multiplicity 4 given by $x_1^2 = x_2^2 = 0$. If $a_1 = a_2$, the pencil is trivial.

\item[(2)] 
\[ 
Q_0 = x_1^2 + x_2^2, \quad Q_1 =  2x_1x_2
\]
The base locus consists of two double lines.
\end{enumerate}
\item[(III. 2)] A plane-pair and a double plane.
\[
Q_0 = 2x_1x_2, \quad Q_1 = x_1^2.
\]
The base locus is a plane with an embedded double line.
\item[(III. 3)]  Two double planes.
\[
Q_0 = x_1^2, \quad Q_1 = x_2^2.
\]
The base locus is a line of multiplicity 4.
\end{itemize}

\begin{lemma}\label{pencil}
Suppose $X$ is a quasi-projective surface over an algebraically closed field $\mathbb{k}$ with only isolated singularities. Assume that the germ of each singular point is the complete intersection of two quadrics. Then $\mathrm{Hilb}^d(X)$ is reducible for sufficiently large $d$ if $\abs{L}$ is from the following cases: I. 1(2, 3, 4), I. 2(1, 2, 3, 4), I. 3(1, 2), I. 4(1, 2), II. 1(1, 2), II. 2, II. 3(1), III. 1(1, 2), III. 3.
\begin{proof}
Let $p$ be a singular point of $X$. Then the germ $(X, p)$ is analytically isomorphic to the affine cone over the base locus of the pencil $\abs{L}$ of quadrics. If $\abs{L}$ is reducible, then $X$ is also reducible locally at $p$, and $\mathrm{Hilb}^d(X)$ is reducible for any $d$. This is the case of I. 1(2, 3), I. 2(2, 3, 4), I. 3(2), I. 4(1, 2), II. 1(1, 2(a)), II. 2, II. 3(1), and III. 1(2). If $\abs{L}$ contains a singular and reduced space curve, then $X$ has a one-dimensional singular locus. By Lemma \ref{easy}, $\mathrm{Hilb}^d(X)$ is reducible for any $d \geq 2$. This proves the cases of I. 2(1), I. 3(1). If $\abs{L}$ contains a non-reduced curve component, then $X$ is non-reduced in a neighborhood of $p$, and $\mathrm{Hilb}^d(X)$ is reducible for any $d \geq 2$. This proves the cases of I. 1(4), II. 1(2(b)), III. 1(1), III. 3. 
\end{proof}
\end{lemma}

Note that the pencils in cases II. 3(2) and III. 2 have common linear factors, and their base loci contain a two-dimensional component. The only remaining case is I. 1(1), in which $\mathrm{Bs}(L)$ is a smooth elliptic curve. Essentially this is the only interesting case. 

\begin{lemma}\label{seven}
Suppose $\abs{L}$ is a pencil of quadrics in $\mathbb{P}^3$.
\begin{itemize}
\item[(1)] Except for the pencils in cases II. 3(2) and III. 2. There exists a seven dimensional linear system of quadrics $\abs{\tilde{L}}$ in $\mathbb{P}^3$ which satisfies the following properties:
\begin{itemize}
\item[(a)] $\abs{L} \subset \abs{\tilde{L}}$.
\item[(b)] $\abs{\tilde{L}}$ generates the space of cubic forms $S_3 = \mathbb{P}(H^0(\mathbb{P}^3, \mathcal{O}_{\mathbb{P}^3}(3)))$.
\end{itemize}
\item[(2)] The three dimensional orthogonal complement $\tilde{L}^{\perp} \subset S_2^* = \mathbb{P}(H^0(\mathbb{P}^3, \mathcal{O}_{\mathbb{P}^3}(2)))^*$ under the pairing (\ref{pairing}) contains three quadratic forms for which the Salmon-Turnbull Pfaffian is non-zero. 
\end{itemize}
\end{lemma}
\begin{proof}
Keep the notations in the list of the base loci of pencils from the preceding section. In each case, we list the seven quadrics that span $\abs{\tilde{L}}$. Hence part (1)(a) will be clear. Part (1)(b) can also be checked in each case. To check part (2), we also list the three quadrics from $\tilde{L}$ whose associated Pfaffian can be calculated explicitly. In the cases I. 2(1, 2, 3, 4), I. 3(1, 2), I. 4(1, 2), the choices of the three quadrics involve homogeneous parameters $[\alpha: \beta] \in \mathbb{P}^1$, and we verify the non-vanishing of the Pfaffian for some general choice of $[\alpha: \beta]$. 

\begin{enumerate}

\item[(I. 1(1))]
\begin{align*}
& \abs{\tilde{L}} = (Q_0, \quad Q_1, \quad x_1x_2, \quad x_1x_3 - x_2x_4, \quad x_1x_4, \quad x_2x_3, \quad x_3y_4) \\ 
& \tilde{L}^{\perp} = \left\{y_1y_3 + y_2y_4, \quad y_1^2 + \dfrac{a_4 - a_1}{a_3 - a_4}y_3^2 + \dfrac{a_3 - a_1}{a_4 - a_3}y_4^2, \quad y_2^2 + \dfrac{a_4 - a_2}{a_3 - a_4}y_3^2 + \dfrac{a_3 - a_2}{a_4 - a_3}y_4^2 \right\} 
\end{align*} 

\item[(I. 1(2))]
\begin{align*}
& \abs{\tilde{L}} = (Q_0, \quad Q_1, \quad x_1x_2, \quad x_1x_3 - x_2x_4, \quad x_1x_4, \quad x_2x_3, \quad x_3x_4) \\ 
& \tilde{L}^{\perp} = \left\{y_1y_3 + y_2y_4, \quad y_1^2 + \dfrac{a_4 - a_1}{a_3 - a_4}y_3^2 + \dfrac{a_3 - a_1}{a_4 - a_3}y_4^2, \quad y_2^2 + \dfrac{a_4 - a_1}{a_3 - a_4}y_3^2 + \dfrac{a_3 - a_1}{a_4 - a_3}y_4^2 \right\} 
\end{align*} 

\item[(I. 1(3))]
\begin{align*}
& \abs{\tilde{L}} = (Q_0, \quad Q_1, \quad x_1x_2, \quad x_1x_3 - x_2x_4, \quad x_1x_4, \quad x_2x_3, \quad x_3x_4) \\ 
& \tilde{L}^{\perp} = \left\{y_1y_3 + y_2y_4, \quad y_1^2 - y_2^2, \quad y_3^2 - y_4^2 \right\}
\end{align*} 

\item[(I. 1(4))]
\begin{align*}
& \abs{\tilde{L}} = (Q_0, \quad Q_1, \quad x_1x_2, \quad x_1x_3 - x_2x_4, \quad x_1x_4, \quad x_2x_3, \quad x_3x_4) \\ 
& \tilde{L}^{\perp} = \left\{y_1y_3 + y_2y_4, \quad a_4y_1^2 - a_1y_4^2, \quad y_2^2 - y_3^2 \right\}
\end{align*} 

\item[(I. 2(1))]
\begin{align*}
& \abs{\tilde{L}} = (Q_0, \quad Q_1, \quad x_1x_3 - x_2x_4, \quad x_1x_4, \quad x_2x_3, \quad x_3x_4, \quad x_1^2) \\ 
& \tilde{L}^{\perp} = \left\{y_1y_3 + y_2y_4, \quad F(\alpha_1, \beta_1), \quad F(\alpha_2, \beta_2) \right\}
\end{align*} 
where $F(\alpha, \beta) = -(\alpha + \beta) y_1y_2 + [(a_1 - a_2)\alpha +(a_1 - a_3)\beta] y_2^2 + \alpha y_3^2 + \beta y_4^2$ for a non-empty open subset of $[\alpha: \beta] \in \mathbb{P}^1$, and $[\alpha_1: \beta_1], [\alpha_2: \beta_2]$ can be any two general distinct points of this open set. 

\item[(I. 2(2))]
\begin{align*}
& \abs{\tilde{L}} = (Q_0, \quad Q_1, \quad x_1x_3 - x_2x_4, \quad x_1x_4, \quad x_2x_3, \quad x_3x_4, \quad x_1^2) \\ 
& \tilde{L}^{\perp} = \left\{y_1y_3 + y_2y_4, \quad F(\alpha_1, \beta_1), \quad F(\alpha_2, \beta_2) \right\}
\end{align*} 
where $F(\alpha, \beta) = -(\alpha + \beta) y_1y_2 + (a_1 - a_3)\beta y_2^2 + \alpha y_3^2 + \beta y_4^2$ for a non-empty open subset of $[\alpha: \beta] \in \mathbb{P}^1$, and $[\alpha_1: \beta_1], [\alpha_2: \beta_2]$ can be any two general distinct points of this open set. 

\item[(I. 2(3))]
\begin{align*}
& \abs{\tilde{L}} = (Q_0, \quad Q_1, \quad x_1x_3 - x_2x_4, \quad x_1x_4, \quad x_2x_3, \quad x_3x_4, \quad x_1^2) \\ 
& \tilde{L}^{\perp} = \left\{y_1y_3 + y_2y_4, \quad F(\alpha_1, \beta_1), \quad F(\alpha_2, \beta_2) \right\}
\end{align*} 
where $F(\alpha, \beta) = -(\alpha + \beta) y_1y_2 + (a_1 - a_2)(\alpha + \beta) y_2^2 + \alpha y_3^2 + \beta y_4^2$ for a non-empty open subset of $[\alpha: \beta] \in \mathbb{P}^1$, and $[\alpha_1: \beta_1], [\alpha_2: \beta_2]$ can be any two general distinct points of this open set. 

\item[(I. 2(4))]
\begin{align*}
& \abs{\tilde{L}} = (Q_0, \quad Q_1, \quad x_1x_3 - x_2x_4, \quad x_1x_4, \quad x_2x_3, \quad x_3x_4, \quad x_1^2) \\ 
& \tilde{L}^{\perp} = \left\{y_1y_3 + y_2y_4, \quad F(\alpha_1, \beta_1), \quad F(\alpha_2, \beta_2) \right\}
\end{align*} 
where $F(\alpha, \beta) = -(\alpha + \beta) y_1y_2 + \alpha y_3^2 + \beta y_4^2$ for a non-empty open subset of $[\alpha: \beta] \in \mathbb{P}^1$, and $[\alpha_1: \beta_1], [\alpha_2: \beta_2]$ can be any two general distinct points of this open set. 

\item[(I. 3(1))]
\begin{align*}
& \abs{\tilde{L}} = (Q_0, \quad Q_1, \quad x_1x_2 - x_3x_4, \quad x_1x_4, \quad x_2x_4, \quad x_1^2, \quad x_3^2) \\
& \tilde{L}^{\perp} = \left\{y_1y_2 + y_3y_4, \quad F(\alpha_1, \beta_1), \quad F(\alpha_2, \beta_2) \right\}
\end{align*} 
where $F(\alpha, \beta) = -2(\alpha + \beta) y_1y_3 + \beta(a_1 - a_2) y_2y_3 + \alpha y_2^2 + \beta y_4^2$ for a non-empty open subset of $[\alpha: \beta] \in \mathbb{P}^1$, and $[\alpha_1: \beta_1], [\alpha_2: \beta_2]$ can be any two general distinct points of this open set.

\item[(I. 3(2))]
\begin{align*}
& \abs{\tilde{L}} = (Q_0, \quad Q_1, \quad x_1x_2 - x_3x_4, \quad x_1x_4, \quad x_2x_4, \quad x_1^2, \quad x_3^2) \\
& \tilde{L}^{\perp} = \left\{y_1y_2 + y_3y_4, \quad F(\alpha_1, \beta_1), \quad F(\alpha_2, \beta_2)\right\}
\end{align*} 
where $F(\alpha, \beta) = -2(\alpha + \beta) y_1y_3 + \alpha y_2^2 + \beta y_4^2$ for a non-empty open subset of $[\alpha: \beta] \in \mathbb{P}^1$, and $[\alpha_1: \beta_1], [\alpha_2: \beta_2]$ can be any two general distinct points of this open set.

\item[(I. 4(1))]
\begin{align*}
& \abs{\tilde{L}} = (Q_0, \quad Q_1, \quad x_1x_3 - x_2x_4, \quad x_1x_4, \quad x_2x_3, \quad x_1^2, \quad x_3^2) \\
& \tilde{L}^{\perp} = \left\{ y_1y_3 + y_2y_4,  \quad F(\alpha_1, \beta_1), \quad F(\alpha_2, \beta_2)\right\}
\end{align*} 
where $F(\alpha, \beta) = \alpha y_1y_2 - \alpha y_3y_4 + \beta y_2^2 -  [(a_1 - a_2)\alpha + \beta]y_4^2$ for a non-empty open subset of $[\alpha: \beta] \in \mathbb{P}^1$, and $[\alpha_1: \beta_1], [\alpha_2: \beta_2]$ can be any two general distinct points of this open set.

\item[(I. 4(2))]
\begin{align*}
& \abs{\tilde{L}} = (Q_0, \quad Q_1, \quad x_1x_3 - x_2x_4, \quad x_1x_4, \quad x_2x_3, \quad x_1^2, \quad x_3^2) \\
& \tilde{L}^{\perp} = \left\{ y_1y_3 + y_2y_4,  \quad F(\alpha_1, \beta_1), \quad F(\alpha_2, \beta_2)\right\}
\end{align*} 
where $F(\alpha, \beta) = \alpha y_1y_2 - \alpha y_3y_4 + \beta y_2^2 - \beta y_4^2$ for a non-empty open subset of $[\alpha: \beta] \in \mathbb{P}^1$, and $[\alpha_1: \beta_1], [\alpha_2: \beta_2]$ can be any two general distinct points of this open set.

\item[(II. 1(1))]
\begin{align*}
& \abs{\tilde{L}} = (Q_0, \quad Q_1, \quad x_1x_2 - x_3x_4, \quad x_1x_3 - x_2x_4, \quad x_1x_4, \quad x_2x_3, \quad x_4^2) \\
& \tilde{L}^{\perp} = \left\{ y_1y_2 + y_3y_4, \quad y_1y_3 + y_2y_4,  \quad y_1^2 + \alpha y_2^2 - (1 + \alpha) y_3^2 \right\}
\end{align*} 
where $\alpha = \dfrac{a_1 - a_3}{a_3 - a_2}$.

\item[(II. 1(2))]
\begin{align*}
& \abs{\tilde{L}} = (Q_0, \quad Q_1, \quad x_1^2, \quad x_1x_4 - x_2x_3, \quad x_1x_3 - x_2x_4, \quad x_3x_4, \quad x_4^2) \\
& \tilde{L}^{\perp} = \left\{ y_1y_4 + y_2y_3, \quad y_1y_3 + y_2y_4,  \quad y_1y_2 - y_3^2 + (a_2 - a_1)y_2^2 \right\}
\end{align*} 

\item[(II. 2)]
\begin{align*}
& \abs{\tilde{L}} = (Q_0, \quad Q_1, \quad x_1x_3 - x_2x_4, \quad x_1x_4, \quad 2x_3x_4 - x_4^2, \quad x_1^2, \quad x_2^2) \\
& \tilde{L}^{\perp} = \left\{ y_1y_3 + y_2y_4,   \quad y_1y_2 - y_3^2, \quad y_3y_4 + y_4^2 \right\}
\end{align*}

\item[(II. 3(1))]
\begin{align*}
& \abs{\tilde{L}} = (Q_0, \quad Q_1, \quad x_1x_3 - x_2x_4, \quad x_1x_4 - x_2x_3, \quad x_3x_4, \quad x_1^2, \quad x_4^2) \\
& \tilde{L}^{\perp} = \left\{ y_1y_3 + y_2y_4,   \quad y_1y_4 + y_2y_3, \quad a_1y_1y_2 - y_2^2 + y_3^2 \right\}
\end{align*}

\item[(II. 3(2))]
\begin{align*}
& \abs{\tilde{L}} = (Q_0, \quad Q_1, \quad x_1x_4, \quad x_2x_3, \quad x_3x_4, \quad x_3^2, \quad 2x_3x_4 - x_4^2) \\
& \tilde{L}^{\perp} = \left\{ y_1y_3 + y_2y_4,   \quad y_1^2 - y_2^2, \quad y_3y_4 + y_4^2 \right\}
\end{align*}

\item[(III. 1(1))]
\begin{align*}
& \abs{\tilde{L}} = (Q_0, \quad Q_1, \quad x_1x_3 - x_2x_4, \quad x_1x_4 - x_2x_3, \quad x_3x_4, \quad x_3^2 - x_4^2, \quad x_1x_2) \\
& \tilde{L}^{\perp} = \left\{ y_1y_3 + y_2y_4,   \quad y_1y_4 + y_2y_3, \quad y_3^2 + y_4^2 \right\}
\end{align*}

\item[(III. 1(2))]
\begin{align*}
& \abs{\tilde{L}} = (Q_0, \quad Q_1, \quad x_1x_3 - x_2x_4, \quad x_1x_4 - x_2x_3, \quad x_3x_4, \quad x_3^2, \quad x_4^2) \\
& \tilde{L}^{\perp} = \left\{ y_1y_3 + y_2y_4, \quad y_1y_4 + y_2y_3, \quad y_1^2 - y_2^2 \right\}
\end{align*}

\item[(III. 2)]
\begin{align*}
& \abs{\tilde{L}} = (Q_0, \quad Q_1, \quad x_1x_3 - x_2x_4, \quad x_2x_3, \quad x_3x_4, \quad x_2^2 + x_3^2, \quad 2x_1x_4 - x_4^2) \\
& \tilde{L}^{\perp} = \left\{ y_1y_3 + y_2y_4,   \quad y_2^2 - y_3^2, \quad y_1y_4 + y_4^2 \right\}
\end{align*}

\item[(III. 3)]
\begin{align*}
& \abs{\tilde{L}} = (Q_0, \quad Q_1, \quad x_1x_3 - x_2x_4, \quad 2x_1x_2 + x_3^2, \quad 2x_3x_4 - x_4^2, \quad x_1x_4, \quad x_2x_3) \\
& \tilde{L}^{\perp} = \left\{ y_1y_3 + y_2y_4,   \quad y_1y_2 - y_3^2, \quad y_3y_4 + y_4^2 \right\}
\end{align*}
\end{enumerate}
\end{proof}

\begin{prop}\label{twisted}
Suppose $\abs{L}$ is a pencil of quadrics in $\mathbb{P}^3$ such that the base locus is a nonsingular quartic elliptic curve $C$ and $X$ is the affine cone over $C$ in $\mathbb{A}^4$. Then the Hilbert scheme of $8$ points $\mathrm{Hilb}^8(X)$ is reducible. 
\end{prop}
\begin{proof}
It suffices to find a length 8 non-smoothable subscheme of $X$. Let $p$ be the vertex of $X$. Then the ideal $I_p \subset \mathbb{k}\llbracket x_1, x_2, x_3, x_4 \rrbracket$ of the germ $(X, p)$ is defined by the pencil $\abs{L}$. By Lemma \ref{seven}(2), there exists a non-smoothable length 8 subscheme $Z$ of $X$ supported at $p$. 
\end{proof}

\begin{prop}
Suppose $X$ is a quasi-projective surface over an algebraically closed field $\mathbb{k}$ of characteristic 0 with only isolated singularities. Assume that at each singular point the embedding dimension of $X$ is exactly 4, such that the tangent cone at each singularity is cut out by a pencil of quadrics. Then the Hilbert scheme of $8$ points $\mathrm{Hilb}^8(X)$ is reducible. 
\end{prop}
\begin{proof}
It suffices to find a length 8 non-smoothable subscheme of $X$. Suppose $p \in X$ is a singular point and suppose the germ $(X, p)$ is locally defined by the ideal $I_p \subset \mathbb{k}\llbracket x_1, x_2, x_3, x_4 \rrbracket$. By assumption, the associated graded ideal $\mathrm{gr}(I_p)$ is generated in degree 2. Then the cases II. 3(2) and III. 2 are impossible. By Lemma \ref{seven}(2) and Proposition \ref{twisted}, there exists a non-smoothable length 8 subscheme $Z$ of the tangent cone at $p$ whose ideal $I_Z$ is generated by seven quadrics and homogeneous polynomials of higher degrees. By Lemma \ref{seven}(1), $I_Z$ is generated exactly by the seven quadrics, and $Z$ is a subscheme of $X$ supported at $p$. 
\end{proof}

\section{Nets of Quadrics}
Let $\abs{L} = (\lambda Q_0 + \mu Q_1 + \nu Q_2)$ be a net of quadrics. The discriminant locus $\Delta$ restricted to the net becomes a plane quartic. In \cite{W78}, Wall provided, for each possible $\Delta_{\abs{L}}$, the list of $\mathrm{SL}(4, \mathbb{C})$-orbits of nets. 

\textbf{Case 1}. If the net is \textbf{stable}. There are two subclasses (class I and class II), depending on whether there exists a singular subpencil.
\begin{itemize}
\item[(I. 1)] Suppose $\Delta_{\abs{L}}$ is irreducible with ordinary double point singularities but no linear components, and there is no singular subpencil in the net.
\begin{enumerate}
\item $Q_0, Q_1, Q_2$ are all plane-pairs: 
\[
Q_0 = 2x_1x_2, \quad Q_1 = 2x_3x_4, \quad Q_2 = (a_1x_1 + a_2x_2 + a_3x_3 + a_4x_4)(b_1x_1 + b_2x_2 + b_3x_3 + b_4x_4),
\]
where $a_3b_4 \neq a_4b_3$, $a_1b_2 \neq a_2b_1$. Then $\mathrm{Bs}(L)$ is a zero-dimensional scheme of length $8$ supported at at least four distinct points with multiplicity at each support at most two.
\item $Q_0, Q_1, Q_2$ are two plane-pairs and a cone:
\[
Q_0 = x_1^2 + 2x_2x_3, \quad Q_1 = 2x_1(a_1x_1 + a_2x_2 + a_3x_3 + a_4x_4), \quad Q_2 = 2x_4(b_1x_1 + b_2x_2 + b_3x_3),
\]
where $a_4 \neq 0$, $b_2^2 + 2b_1b_3 \neq 0$, and $b_2a_3 \neq b_3a_2$. Then $\mathrm{Bs}(L)$ is a zero-dimensional scheme of length $8$ with multiplicity two at the vertex of the cone together with six distinct reduced points. 
\item $Q_0, Q_1, Q_2$ are one plane-pair and two cones:
\[ 
Q_0 = x_2^2 + 2x_1x_3, \quad Q_1 = x_3^2 + 2x_2x_4, \quad Q_2 = (a_1x_1 + a_2x_2 + a_3x_3)(b_2x_2 + b_3x_3 + b_4x_4),
\]
where $a_1 \neq 0$, $b_4 \neq 0$, $a_2^2 + 2a_1a_3 \neq 0$, and $b_3^2 + 2b_2b_4 \neq 0$. Then $\mathrm{Bs}(L)$ is a zero-dimensional scheme of length $8$ with multiplicity at most two at any point of its support.
\item $Q_0, Q_1, Q_2$ are three cones:
\begin{align*}
& Q_0 = x_2^2 + 2x_1x_3, \quad Q_1 = x_3^2 + 2x_2x_4, \\
& Q_2 = c(x_2 - x_3)^2 + 2f(2x_4 + x_3)(x_2 - x_3) + 2g(2x_1 + x_2)(x_3 - x_2) + 2h(2x_1 + x_2)(2x_4 + x_3),
\end{align*}
where $h \neq 0$ and $ch + 2fg \neq 0$. Then $\mathrm{Bs}(L)$ is a zero-dimensional scheme of length $8$.
\end{enumerate}
\item[(I. 2)] If $\Delta_{\abs{L}}$ has a higher double point and there is no singular pencil. In all of the three cases below, $a_{11} \neq 0$ otherwise the net is not stable. 
\begin{enumerate}
\item $Q_2$ is a cone, then this is the net of a quadrics through a twisted cubic curve in $\mathbb{P}^3$.
\[ 
Q_0 = a_{11}x_1^2 + a_{22}x_2^2 + a_{33}x_3^2 + 2x_2x_4, \quad Q_1 = 2b_{12}x_1x_2 + b_{22}x_2^2 + b_{33}x_3^2 + 2x_3x_4, \quad Q_2 = x_2^2 + 2x_1x_3.
\]
\item $Q_2$ is a plane-pair and the pencil $(\lambda Q_0 + \mu Q_1)$ is of type $(3, 1)$.
\[
Q_0 = a_{11}x_1^2 + 2x_3x_4, \quad Q_1 = b_{11}x_1^2 + b_{13}x_1x_3 + x_1x_4 + b_{23}x_2x_3 + cx_3^2, \quad Q_2 = 2x_1x_2, 
\]
where $c \neq 0$. Then $\mathrm{Bs}(L)$ is the union of a plane cubic curve and a line.
\item $Q_2$ is a plane-pair and the pencil $(\lambda Q_0 + \mu Q_1)$ is of type $(2, 2)$.
\[ 
Q_0 = a_{11}x_1^2 + 2x_3x_4, \quad Q_1 = b_{11}x_1^2 + b_{22}x_2^2 + x_1x_3 + b_{14}x_1x_4 + b_{23}x_2x_3 + b_{24}x_2x_4, \quad Q_2 = 2x_1x_2.
\]
Then $\mathrm{Bs}(L)$ is a zero-dimensional scheme of length 8 with multiplicity two at $[0 \colon 0 \colon 1 \colon 0]$ and at $[0 \colon 0 \colon 0 \colon 1]$ union with six other distinct points.
\end{enumerate}
\item[(I. 3)] If $\Delta_{\abs{L}}$ has a triple point and there is no singular pencil.
\begin{enumerate}
\item $Q_2$ is a plane-pair. 
\[ 
Q_0 = a_{11}x_1^2 + a_{22}x_2^2 + a_{33}x_3^2 + 2x_1x_4, \quad Q_1 = b_{11}x_1^2 + b_{22}x_2^2 + b_{13}x_1x_3 + b_{33}x_3^2 + 2x_2x_4, \quad Q_2 = 2x_1x_2,
\]
where $a_{33} \neq 0$. Then $\mathrm{Bs}(L)$ is a zero-dimensional scheme of length 8.
\item $Q_2$ is a repeated plane. There are three subcases:
\begin{align*}
& Q_0 = x_1^2 + x_3^2 + a_{24}x_2x_4 + a_{34}x_3x_4 + a_{44}x_4^2, \quad Q_1 = x_2^2 + b_{14}x_1x_4 + x_3^2 - a_{34}x_3x_4, \quad Q_2 = x_4^2; \\
& Q_0 = 2x_1x_3 + x_2^2 + a_{44}x_4^2, \quad Q_1 = 2x_2x_3 + b_{14}x_1x_4 + b_{24}x_2x_4 + b_{34}x_3x_4, \quad Q_2 = x_4^2; \\
& Q_0 = 2x_1x_2 + a_{34}x_3x_4 + a_{44}x_4^2, \quad Q_1 = x_2^2 + b_{14}x_1x_4 + x_3^2 + b_{24}x_2x_4, \quad Q_2 = x_4^2.
\end{align*}
In each of the three cases above, $\mathrm{Bs}(L)$ is a zero-dimensional scheme of length 8 supported at at most four distinct points with even multiplicities at each of the support.
\end{enumerate}
\item[(II. 1)] There is a singular subpencil, and the singular subpencil has a common vertex. Then the net is reduced to $Q_0 = x_1^2 + Q_0'$, and none of $Q_0'$, $Q_1$, or $Q_2$ involves $x_1$. The classification is equivalent to the classification of nets of conics. 
\begin{enumerate}
\item (Type A, B, B*, and C) The discriminant of the planar net is irreducible.
\[
Q_0 = x_1^2 + 2x_2x_4 + x_3^2, \quad Q_1 = 2x_3x_4, \quad Q_2 = x_2^2 + 2gx_3^2 - cx_4^2 - 2gx_2x_4
\]
for some $c$ and $g$.
\item (Type D)
\[
Q_0 = x_1^2 + x_2^2, \quad Q_1 = x_3^2, \quad Q_2 = x_4^2 + 2x_2x_4
\]
\item (Type D*)
\[
Q_0 = x_1^2 + 2x_2x_4, \quad Q_1 = 2x_3x_4, \quad Q_2 = x_4^2 + 2x_2x_3
\]
\item (Type E)
\[
Q_0 = x_1^2 + x_2^2, \quad Q_1 = x_3^2, \quad Q_2 = x_4^2
\]
\item (\textcolor{red}{Type E*})
\[
Q_0 = x_1^2 + 2x_2x_4, \quad Q_1 = 2x_3x_4, \quad Q_2 = 2x_2x_4
\]
\item (Type F)
\[
Q_0 = x_1^2 + x_2^2, \quad Q_1 = x_3^2, \quad Q_2 = 2x_4(x_2 + x_3)
\]
\item (Type F*)
\[
Q_0 = x_1^2 + x_2^2, \quad Q_1 = 2x_2x_3, \quad Q_2 = x_3^2 + x_4^2
\]
\item (Type G)
\[
Q_0 = x_1^2 + x_2^2, \quad Q_1 = x_3^2, \quad Q_2 = 2x_3x_4
\]
\item (Type G*)
\[
Q_0 = x_1^2 + x_2^2, \quad Q_1 = 2x_2x_3, \quad Q_2 = 2x_3x_4
\]
\item (\textcolor{red}{Type H})
\[
Q_0 = x_1^2 + x_2^2, \quad Q_1 = 2x_2x_4, \quad Q_2 = x_3^2 + 2x_2x_4
\]
\item (Type I)
\[
Q_0 = x_1^2 + x_2^2, \quad Q_1 = 2x_2x_3, \quad Q_2 = x_3^2
\]
\item (Type I*)  
\[
Q_0 = x_1^2 + 2x_2x_4, \quad Q_1 = 2x_3x_4, \quad Q_2 = x_4^2
\]       
\end{enumerate}
The cases where $\mathrm{Bs}(L)$ contains at least a one-dimensional component are II. 1(5, 8, 9, 11, 12). The positive dimensional component of $\mathrm{Bs}(L)$ is a plane conic, which is either a double line (II. 1(5, 11, 12)) or the union of two reduced lines (II. 1(8, 9)). Except for case II. 1(9), $\mathrm{Bs}(L)$ has embedded points, which is the intersection of the two line in case II. 1(8). In all remaining cases, $\mathrm{Bs}(L)$ is zero-dimensional.

\item[(II. 2)] There is a singular subpencil. Any singular subpencil does not have a common vertex. Then $Q_0$ and $Q_1$ can be normalized as
\[
Q_0 = x_1^2 + 2x_2x_3, \quad Q_1 = 2x_2x_4.
\]
The possible normal forms for $Q_2$ can be determined in the following 4 cases. In all of the four cases, the net has a zero-dimensional base locus.
\begin{enumerate}
\item If $a_{33} \neq 0, a_{33}a_{44} - a_{34}^2 \neq 0$,
\[
Q_2 = a_{22}x_2^2 + a_{33}x_3^2 + a_{44}x_4^2 + a_{12}x_1x_2 + a_{14}x_1x_4 + a_{23}x_2x_3 + a_{34}x_3x_4,
\]
where all of the coefficients can be uniquely solved in terms of $a_{14}$ if $a_{14} \neq 0$ or in terms of $a_{12}$ if $a_{14} = 0$.
\item If $a_{33} \neq 0, a_{33}a_{44} - a_{34}^2 = 0$,
\[
Q_2 = a_{22}x_2^2 + a_{33}x_3^2  + a_{12}x_1x_2 + a_{14}x_1x_4 + a_{23}x_2x_3,
\]
where $a_{14} \neq 0$ and all of the coefficients can be uniquely solved in terms of $a_{14}$.
\item If $a_{33} = 0, a_{34}^2 \neq 0$,
\[
Q_2 = a_{11}x_1^2 + a_{22}x_2^2 + a_{13}x_1x_3 + a_{23}x_2x_3 + a_{34}x_3x_4,
\]
where all of the coefficients can be uniquely solved in terms of $a_{13}$ if $a_{13} \neq 0$ or in terms of $a_{11}$ if $a_{13} = 0$.
\item If $a_{33} = 0$ and $a_{34}^2 = 0$,
\[
Q_2 = a_{22}x_2^2 + a_{44}x_4^2 + a_{12}x_1x_2 + a_{13}x_1x_3 + a_{14}x_1x_4 + a_{23}x_2x_3,
\]
where $a_{13} \neq 0$ and all of the coefficients can be uniquely solved in terms of $a_{13}$.
\end{enumerate}
\end{itemize}

\textbf{Case 2}. If the net is \textbf{unstable but semistable}. There are two types of nets according to the two types of possible symmetric matrices of unstable but semistable quadrics.
\begin{itemize}
\item[(III. 1)] Type $(s = 2)$: nets of quadrics with a common line $(x_3 = x_4 = 0)$.
\[
Q_0 = 2x_1x_4 + a_{33}x_3^2, \quad Q_1 = 2x_2x_3 + b_{44}x_4^2, \quad Q_2 = 2x_2x_4 + c_{33}x_3^2 + 2c_{34}x_3x_4 + c_{44}x_4^2
\]
If $a_{33}, b_{44}, c_{33}, c_{34}$, and $c_{44}$ do not vanish simultaneously, the line $(x_3 = x_4 = 0)$ has two embedded points. If $a_{33} = b_{44} = c_{33} = c_{34} = c_{44} = 0$, there exists another line $(x_1 = x_2 = 0)$ in $\mathrm{Bs}(L)$.

\item[(III. 2)] Type $(s = 1)$: there are 10 classes. In the first three classes $\Delta_{\abs{L}}$ has no quadruple point. The remaining classes are those in which $\Delta_{\abs{L}}$ has a quadruple point.
\begin{itemize}
\item[(1)] $\Delta_{\abs{L}}$ is a conic with repeated chord.
\[
Q_0 = x_2^2 + 2a_{34}x_3x_4  + a_{44}x_4^2, \quad Q_1 = 2x_1x_4 + 2 x_2x_3, \quad Q_2 = x_3^2 + 2b_{24}x_2x_4 + b_{44}x_4^2 
\]
\item[(2)] $\Delta_{\abs{L}}$ is a line-pair with repeated chord.
\[
Q_0 = x_2^2 + 2a_{34}x_3x_4  + a_{44}x_4^2, \quad Q_1 = 2x_1x_4, \quad Q_2 = x_3^2 + 2b_{24}x_2x_4 + b_{44}x_4^2 
\]
\item[(3)] $\Delta_{\abs{L}}$ is a conic with repeated tangent. 
\[
Q_0 = x_2^2 + 2a_{24}x_2x_4 + a_{44}x_4^2, \quad Q_1 = 2x_1x_4 + x_3^2, \quad Q_2 = 2x_2x_3 + b_{44}x_4^2 
\]
\item[(4)] $\Delta_{\abs{L}}$ is a repeated line-pair.
\[
Q_0 = x_2^2 + 2a_{24}x_2x_4 + a_{44}x_4^2, \quad Q_1 = 2x_1x_4, \quad Q_2 = 2x_2x_3 + b_{44}x_4^2 
\]
\item[(5)] $\Delta_{\abs{L}}$ consists of three concurrent lines with one repeated.
\[
Q_0 = x_2^2 + x_3^2 + 2x_1x_4, \quad Q_1 = 2x_2x_3 + a_{44}x_4^2,  \quad Q_2 = 2x_2x_4 + 2b_{34}x_3x_4 + b_{44}x_4^2 
\]
\item[(6)] $\Delta_{\abs{L}}$ is a repeated line-pair.
\[
Q_0 = x_2^2 + 2x_1x_4, \quad Q_1 = 2x_2x_3 + a_{44}x_4^2,  \quad Q_2 = 2x_2x_4 + 2b_{34}x_3x_4 + b_{44}x_4^2 
\]
\item[(7)] $\Delta_{\abs{L}}$ is a repeated line-pair.
\[
Q_0 = 2x_1x_4, \quad Q_1 = 2x_2x_3 + a_{44}x_4^2,  \quad Q_2 = 2x_2x_4 + 2b_{34}x_3x_4 + b_{44}x_4^2 
\]
\item[(8)] $\Delta_{\abs{L}}$ consists of a reduced line and a triple line.
\[
Q_0 = x_2^2 + 2a_{34}x_3x_4 + a_{44}x_4^2, \quad Q_1 = 2x_1x_4 + x_3^2,  \quad Q_2 = 2x_2x_4 + 2b_{34}x_3x_4 + b_{44}x_4^2 
\]
\item[(9)] $\Delta_{\abs{L}}$ consists of a fourfold line.
\[
Q_0 = x_2^2 + 2a_{34}x_3x_4 + a_{44}x_4^2, \quad Q_1 = 2x_1x_4 + 2x_2x_3,  \quad Q_2 = 2x_2x_4 + 2b_{34}x_3x_4 + b_{44}x_4^2 
\]
\item[(10)] $\Delta_{\abs{L}}$ consists of a fourfold line.
\[
Q_0 = 2x_2x_4 + a_{44}x_4^2, \quad Q_1 = 2x_1x_4 + 2x_2x_3,  \quad Q_2 = 2x_2x_4 + 2b_{34}x_3x_4 
\]
\end{itemize}
In the cases III. 2(4, 6, 9), $\mathrm{Bs}(L)$ contains a line $(x_2 = x_4 = 0)$ with at least one embedded point. In III. 2(7, 10), $\mathrm{Bs}(L)$ contains two lines $(x_3 = x_4 = 0)$ and $(x_2 = x_4 = 0)$. In all remaining cases, $\mathrm{Bs}(L)$ is zero-dimensional. 
\end{itemize}


\textbf{Case 3}. If the net is \textbf{not semistable}. 
\begin{itemize}
\item[(IV. 1)] 
\[
Q_0 = 2x_1x_4 + a_{33}x_3^2, \quad Q_1 = 2x_2x_4, \quad Q_2 = 2x_3x_4 ( \textrm{or } x_4^2).
\]
where $a_{33} \neq 0$. If $Q_2 = 2x_3x_4$, then $\mathrm{Bs}(L)$ contains a double line $(x_4 = x_3^2 = 0)$ and and an isolated point $[0 \colon 0 \colon 0 \colon 1]$. If $Q_2 = x_4^2$, then $\mathrm{Bs}(L)$ contains a double line $(x_4 = x_3^2 = 0)$.
\item[(IV. 2)] 
\[
Q_0 = 2x_1x_4, \quad Q_1 = 2x_2x_4, \quad Q_2 = x_3^2 ( \textrm{or } x_3^2 + x_4^2).
\]
If $Q_2 = x_3^2$, then $\mathrm{Bs}(L)$ contains a double line $(x_4 = x_3^2 = 0)$ and an isolated thick point $x_1 = x_2 = x_3^2 = 0$. If $Q_2 = x_3^2 + x_4^2$, then $\mathrm{Bs}(L)$ contains a double line $(x_4 = x_3^2 = 0)$ and two reduced isolated points, which are the intersection of the line $x_1 = x_2 = 0$ with the plane-pair $Q_2$. 
\item[(IV. 3)] 
\[
Q_0 = 2x_1x_4, \quad Q_1 = 2x_2x_4, \quad Q_2 = 2x_3x_4 ( \textrm{or } x_4^2).
\]
This is the specialization of IV. 1 with $a_{33} = 0$. In this case, $\mathrm{Bs}(L)$ contains a two-dimensional component $x_4 = 0$ and an isolated point $[0 \colon 0 \colon 0 \colon 1]$. If $Q_2 = 2x_3x_4$, then $\abs{L}$ contains the pencil II. 3(2), and if $Q_2 = x_4^2$, it contains the pencil III. 2. 

\end{itemize}

To summarize, $\mathrm{Bs}(L)$ contains a positive dimensional component in cases I. 2(1, 2), II. 1(5, 8, 9, 11, 12), III. 1, III. 2(4, 6, 7, 9, 10), IV. 1, IV. 2, and IV. 3; in the remaining cases $\mathrm{Bs}(L)$ are zero-dimensional. For the interest of this article, we focus on the former situation. Note that $\mathrm{Bs}(L)$ is reduced and irreducible if and only if $\mathrm{Bs}(L)$ is a twisted cubic curve. Essentially this is the only interesting case. In the same way of proving Lemma \ref{pencil}, we can prove the following

\begin{lemma}\label{net}
Suppose $X$ is a quasi-projective surface over an algebraically closed field $\mathbb{k}$ with only isolated singularities of embedding dimension 4. Assume that the germ of each singular point is the intersection of three quadrics. Then $\mathrm{Hilb}^d(X)$ is reducible for sufficiently large $d$ if $\abs{L}$ is from the following cases: I. 2(2), II. 1(5, 8, 9, 11, 12), III. 1, III. 2(4, 6, 7, 9, 10), IV. 1, and IV. 2. 
\end{lemma}

If $\mathrm{Bs}(L)$ is a twisted cubic curve, one can find a non-smoothable length 8 subscheme on the tangent cone in the same way as in the case where $\abs{L}$ is a pencil with $\mathrm{Bs}(L)$ a smooth elliptic curve. We try to find a seven dimensional linear system $\abs{\tilde{L}}$ of quadrics containing $\abs{L}$ such that there exists three quadratic forms $F_1, F_2$, and $F_3$ in $\tilde{L}^{\perp} \in G(3, S_2^*)$ with non-vanishing Salmon-Turnbull Pfaffian. 

\begin{prop}
Suppose $X$ is a quasi-projective surface over an algebraically closed field $\mathbb{k}$ with only isolated singularities of embedding dimension 4. Assume that there exists a singular point $p \in X$ such that the germ $(X, p)$ is the intersection of three quadrics which define a twisted cubic curve in $\mathbb{P}^3$. Then $\mathrm{Hilb}^8(X)$ is reducible.
\end{prop}
\begin{proof}
The net of quadrics is in the case I. 2(1), for which the normal forms of the quadrics are:
\[ 
Q_0 = a_{11}x_1^2 + a_{22}x_2^2 + a_{33}x_3^2 + 2x_2x_4, \quad Q_1 = 2b_{12}x_1x_2 + b_{22}x_2^2 + b_{33}x_3^2 + 2x_3x_4, \quad Q_2 = x_2^2 + 2x_1x_3.
\]
The three quadratic forms can be chosen as
\[
F_1 = y_1^2 - a_{11}y_2y_4, \quad F_2 = y_1y_4 - y_2y_3, \quad F_3 = y_4^2 - y_3^2 + b_{33}y_3y_4 + a_{33}y_2y_4.
\]
Alternatively, the variety of twisted cubics $Y_0$ is quasi-projective, has dimension 12, which admits a compactification by the component of the Hilbert scheme $\mathrm{Hilb}^{3t + 1}(\mathbb{P}^3)$ containing the points corresponding to twisted cubics. Consider the incidence correspondence $I = \{(C, q)\}$ of pairs where $C$ is a twisted cubic curve and $q: U^* \hookrightarrow G(3, S_2^*)$ is a net of quadratic forms which annihilates the net of quadrics generating the ideal in $\mathbb{P}^3$ defining $C$. Let $p_1: I \to G(3, S_2^*)$ and $p_2: I \to Y_0$ be the two projection maps. By a theorem of Mukai, the fiber of $p_1$ at a \textit{general} net of quadratic forms is a smooth prime Fano 3-fold of genus 12. Hence $I$ is irreducible of dimension 24. On the other hand, the fiber of $p_2$ at any twisted cubic curve is identified with the Grassmannian $G(3, H^0(C, \mathcal{O}_C(2)))$, which has dimension 12. Then the projection $p_2$ is surjective since $Y_0$ is smooth irreducible. Since the non-vanishing of Pfaffian is an open condition, satisfied by the complement of a divisor of $G(3, S_2^*)$, hence for any twisted cubic, there is a net of quadratic forms with non-vanishing Salmon-Turnbull Pfaffian. Therefore, there is a non-smoothable length 8 subscheme on the cone over this twisted cubic. 
\end{proof}

\begin{rmk}(Surfaces defined by more than three quadrics) Suppose $\abs{L}$ is any linear system of quadrics in $\mathbb{P}^3$ of dimension at least 3 such that $\mathrm{Bs}(L)$ is at most one-dimensional and that it contains a one-dimensional component. Then any of the one-dimensional components of $\mathrm{Bs}(L)$ must be either a plane conic or a line. It is known that the Hilbert scheme of points on the affine cone over a plane conic is irreducible. 
\end{rmk}

The remaining cases from the preceding two sections are those where the base loci of the pencil or the net of quadrics contains a two-dimensional component. This component must be linear and generically reduced. If the pencil or net does not have a double plane, then the linear component does not have any embedded point. Also, if $\abs{L}$ is any linear system of quadrics of dimension at least 4, then $\mathrm{Bs}(L)$ is at most one-dimensional. So we are left with the following cases: 

\textit{$(X, p)$ is the germ of a surface singularity of embedding dimension 4, $\abs{L}$ spans $\mathrm{gr}_2(I_p)$, the degree 2 part of the tangent cone, $\dim \abs{L} \leq 3$, and $\mathrm{gr}(I_p)$ requires at least one generator of degree at least 3. }

\begin{lemma}\label{cubic}
Suppose $\abs{L}$ is a linear system of quadrics in $\mathbb{P}^3$ such that $\mathrm{Bs}(L)$ contains a two-dimensional component and $\dim \abs{L} \leq 3$. Then 
\begin{itemize}
\item[(1)] there exists a net of quadratic forms annihilating $\abs{L}$ such that the associated Salmon-Turnbull Pfaffian is non-zero if and only if $\abs{L}$ is a pencil:
\begin{itemize}
\item[(a)] $\abs{L} = (x_1x_2, x_1x_3)$;
\item[(b)] $\abs{L} = (x_1^2, x_1x_2)$.
\end{itemize}
\item[(2)] any net of quadratic forms annihilating $\abs{L}$ has vanishing Salmon-Turnbull Pfaffian if and only if $\abs{L}$ is at least two-dimensional, which can only be one of the following cases.
\begin{itemize}
\item[(a)] $\abs{L} = (x_1x_2, x_1x_3, x_1x_4)$;
\item[(b)] $\abs{L} = (x_1^2, x_1x_2, x_1x_3)$;
\item[(c)] $\abs{L} = (x_1^2, x_1x_2, x_1x_3, x_1x_4)$.
\end{itemize}
\end{itemize}
\end{lemma}
Note that (c) in part (2) is the only case where the web of quadrics has a two-dimensional base locus. 

\begin{proof}
Part (1) is proven in Lemma \ref{seven}. Case (a) is the pencil II. 3(2). Case (b) is the pencil III. 2.

To prove part (2), note that there is a linear common factor $x_1$ in $\abs{L}$ in all of the three cases. Suppose $(F_1, F_2, F_3)$ is a net of quadratic forms that annihilates $\abs{L}$. Then at most one of the $F_i$ contains a term that is divisible by $y_1$. Then in the $12 \times 12$ skew-symmetric matrix associated to $(F_1, F_2, F_3)$ there exists at least one column whose entries are all zeros. Then the Pfaffian is zero. 
\end{proof}

\begin{prop}
Suppose $X$ is a quasi-projective surface with only isolated singularities, and $(X, p)$ is the germ of a singularity of embedding dimension 4 such that the degree 2 part of the tangent cone is spanned by a system of quadrics that belongs to one of the cases of Lemma \ref{cubic}. Then $\mathrm{Hilb}^d(X)$ is reducible for sufficiently large $d$.
\end{prop}
\begin{proof}
Let $\abs{L}$ be the system of quadrics that spans $\mathrm{gr}_2(I_p)$. If $\abs{L}$ is a pencil, then Lemma \ref{cubic} (1) implies that there exists a non-smoothable length 8 subscheme of $X$ supported at $p$. Hence $\mathrm{Hilb}^8(X)$ is reducible. 

If $\abs{L}$ is either in case (a) or (b) of Lemma \ref{cubic} (2), namely a net of case IV. 3, then it is the limit of a family of nets of quadrics of case IV. 1 (by letting the coefficient $a_{33} \to 0$). Let $\{x_1x_2 \textrm{ (or } x_1^2), x_1x_3, x_1x_4, f_1, \dots, f_r \}$ be a minimal generator set of $\mathrm{gr}(I_p)$ where $f_i$ is homogeneous of degree $d_i \geq 3$ for $i = 1, \dots, r$. In either case (a) or (b), there is a flat family of nets of quadrics $\abs{L_t}$ such that
\begin{itemize}
\item[(1)] The net $\abs{L_t}$ generates a height two homogeneous ideal $I_t$ for any $t \neq 0$.
\item[(2)] The flat limit of $I_t$ is the ideal $\mathrm{gr}(I_p)$. 
\end{itemize}
The procedure is to find the family of nets of quadrics with the same Hilbert function as that of $(X, p)$. Since $(X, p)$ is a germ on a surface, the numbers of generators of each degree satisfy some conditions (e.g., there are at most three cubic generators, etc.)

Then there is a family $\{(X_t, p_t)\}$ of germs of surface singularities, where $\mathrm{gr}(I_{p_t}) = I_t$ for any $t \neq 0$ and $(X_0, p_0) = (X, p)$. The Hilbert schemes $\mathrm{Hilb}^d(t)$ is reducible for any $t \neq 0$ and any sufficiently large $d$. Then $\mathrm{Hilb}^d(X)$ is reducible. 


The only case remained is case (c) in part (2) of Lemma \ref{cubic}. This case can be settled using Iarrobino's classical Grassmannian argument. Denote by $R = \mathbb{C}\llbracket x, y, z, w \rrbracket /(x_1^2, x_1x_2, x_1x_3, x_1x_4)$, and the space of homogeneous $d$-forms in $R$ by $R_d$. Note that $\dim R_d = {d + 2 \choose 2}$ for any $d \geq 2$ and $\dim R_1 = 4$. Consider the family of subschemes of length $28$ defined by ideals generated by 7 quartic forms in $R$ and $\mathfrak{m}^5$, where $\mathfrak{m}$ is the maximal ideal of $R$. These subschemes form a $7 \times 8 = 56$-dimensional family. Since the expected dimension of the Hilbert scheme of 28 points on the surface is also 56, the generic member of this family must be non-smoothable. 


\end{proof}

\section{Cubic Hypersurface Singularities}

Suppose $F$ is a general smooth cubic form in $x_1, x_2, x_3$. Denote by $X = \mathrm{Spec}(\mathbb{C}[x_1, x_2, x_3]/(F))$ the affine cone over the smooth plane cubic curve defined by $F$. Denote by $R = \mathbb{C}\llbracket x_1, x_2, x_3 \rrbracket/(F)$ the complete local ring of $X$ at the vertex of the cone. Iarrobino's idea of constructing non-smoothable \textit{compressed algebras} works in this situation. For a local Artinian algebra $(A, \mathfrak{m})$, its \textit{socle type} is the finite sequence of non-negative integers $E(A) = (0, e_1, e_2, \dots, e_j)$, where $e_i = \dim ((0 : \mathfrak{m}) \cap \mathfrak{m}^i) / ((0 : \mathfrak{m}) \cap \mathfrak{m}^{i + 1})$, and $j = \min_k {\mathfrak{m}^k = 0}$. Such a finite sequence is said to be \textit{permissible} if it is the socle type of some local Artinian algebra. The Hilbert function of $A$ is another finite sequence of positive integers $H(A) = (h_0, h_1, \dots)$. The algebra $A$ with socle type $E(A)$ is said to be \textit{compressed} if for any Artinian algebra $B$ with $E(B) = E(A)$ we have $h_i(A) \geq h_i(B)$ for any $i \geq 0$.  The main result of (\cite{I84}) is to understand the families of compressed algebras with prescribed socle types as quotient algebras of some power series ring. 

\begin{lemma}(cf. \cite[Corollary 3.8]{I84})
Suppose $(A, \mathfrak{m})$ is a local Artinian algebra with associated graded algebra $\mathrm{gr}(A)$. 
\begin{itemize}
\item[1.] The algebra $A$ is compressed if and only if $\mathrm{gr}(A)$ is compressed.
\item[2.] If $A$ is compressed whose socle is generated by $f_1, \dots, f_r$. Then the socle of $\mathrm{gr}(A)$ is generated by $F_1, \dots, F_r$ where $F_i$ is the initial form of $f_i$ for $i = 1, \dots, r$.
\end{itemize}
\end{lemma}

The next proposition is a weaker generalization of \cite[Theorem IIB, IIC]{I84} from the case where the Artinian algebras under consideration are quotients of a power series ring to the current  case where the Artinian algebras are quotient algebras of the hypersurface ring $R$. We are only interested in calculating the dimensions of the relevant spaces $G(E)$ and $Z(E)$, but further work is required to better understand their geometry. 

\begin{prop}
Suppose $E$ is a permissible socle type for Artinian quotients of $R = \mathbb{C}\llbracket x_1, x_2, x_3 \rrbracket/(F)$. 
\begin{itemize}
\item[1.]
Denote by $G(E)$ the space that parameterizes compressed Artinian quotients of $R$ with socle type $E$. Then $G(E)$ has dimension 
\[
\dim G(E) = \sum_i e_i(\dim R_i - h_i),
\]
where $h_i$ is the $i$-th component of the Hilbert function of any compressed algebra with socle type $E$.
\item[2.] Denote by $Z(E)$ the space of compressed algebras with socle type $E$. Then $Z(E)$ has dimension
\[
\dim Z(E) = \sum_i\sum_{u \geq i} e_u(\dim R_i - h_i).
\]
\end{itemize}
\end{prop}

% ----------------------------------------------------------------
\bibliographystyle{amsalpha}
\begin{thebibliography}{ShB91a}

\bibitem[CEVV]{CEVV}
Cartwright et al., (2009), Hilbert schemes of 8 points. {\em Algebra Number Theory}, 3(7), 763-795.

\bibitem[EI78]{EI78}
Iarrobino, A., Emsalem, J. (1978), Some zero-dimensional generic singularities: finite algebras having small tangent spaces. {\em Compositio Math.}, 36, 145-188.

\bibitem[HP52]{HP52}
Hodge, W. V. D., Pedoe, D., {\em Methods of Algebraic Geometry}, Vol. 11, Cambridge University Press, 1952.

\bibitem[I84]{I84}
Iarrobino, I. (1984), Compressed algebras: Artin algebras having given socle degrees and maximal length. {\em Trans. AMS}, 285(1), 337-378.

\bibitem[W78]{W78}
Wall, C. T. C. (1978), Nets of quadrics, and theta-characteristics of singular curves. {\em Philos. Trans. Roy. Soc. London Ser. A}, 289, no. 1357, 229-269.

\end{thebibliography}
\end{document}
% ----------------------------------------------------------------
